\section{Conceptos más importantes}

% []
% \tableofcontents
\begin{enumerate}
\item Pruebas de hipótesis.
\item $p-$valores.
\item Distribución normal.
\item Correlación.
\end{enumerate}


\subsection{Muestreo aleatorio y teorema del límite central}
Entender el concepto de \emph{muestreo aleatorio} a través de ejemplos e ilustrar las aplicaciones del \emph{teorema del límite central}. 

  Estos dos conceptos son la columna vertebral de las pruebas de hipótesis.


\subsection{Pruebas de hipótesis}
Entender el significado de los términos tales como \emph{hipótesis nula}, \emph{hipótesis alternativa}, \emph{intervalos de confianza}, \emph{$p-$valores}, \emph{nivel de significación}, etc.
%   Desarrollaremos una guía de la implementación de pruebas de hipótesis, seguidas por un ejemplo.

\subsection{Pruebas $\chi-$cuadrada}
Calcularemos el estadístico $\chi$-cuadrada y describiremos el uso de pruebas $\chi$-cuadrada con un par de ejemplos.

 \subsection{Correlación}
  Entenderemos el significado y la significación de la correlación entre dos variables, de los coeficientes de correlación y calcularemos y visualizaremos la correlación entre variables de una base de datos.
 
