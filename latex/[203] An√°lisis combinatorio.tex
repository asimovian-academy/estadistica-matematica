\section{Análisis combinatorio}

{}
El \emph{análisis combinatorio} es una manera sofisticada de contar.


\subsection{Principio fundamental del conteo y diagramas de árbol}

Si una tarea se puede realizar en $n$ formas diferentes y otra en $m$ formas diferentes, entonces las dos tareas se pueden realizar en $n\times m$ formas diferentes.


{}
\begin{ejemplo}
	\label{exmp:1.14}
\end{ejemplo}
\begin{enumerate}
	\item Si una persona tiene 2 camisas y 4 corbatas, ?`de cuantas formas puede combinarlas?
	\item Construya un diagrama de árbol para representar todas estas opciones.
\end{enumerate}



\subsection{Permutaciones}
{}
Si tenemos $n$ objetos distintos y queremos ordenarlos tendremos
\begin{align}
	n \times (n-1) \times ... 2\times 1
\end{align} formas diferentes de hacerlo.

{}
\begin{defn}[$n$ factorial]
	\begin{equation}
		n! = \begin{cases}
			1 & n=0 \\
			n\times(n-1)! & n>0
		\end{cases}
	\end{equation}
	
\end{defn}


{}
Si tenemos $n$ objetos distintos y queremos arreglar $r$ de estos en una linea, entonces tendremos una \emph{permutación} de $n$ en $r$ dada por
\begin{align}
	\label{1.25}
	P^{n}_{r}=n\times(n-1)\times...\left( n-r+1 \right)
\end{align} 
o de manera equivalente
\begin{align}
	\label{1.27}
	P^{n}_{r}=\dfrac{n!}{(n-r)!}
\end{align}


{}
\begin{ejemplo}
	\label{exmp:1.16}
	?`Cuantas permutaciones de longitud 3 se pueden formar con las letras $A,B,C,D,E,F,G$?
\end{ejemplo}


{}
\begin{ejemplo}
	\label{exmp:1.17}
	Encuentre el número de permutaciones diferentes de las 11 letras de la palabra \emph{MISSISSIPPI}.
\end{ejemplo}


\subsection{Combinaciones}
{}
En una \emph{permutación}, uno está interesado en el orden de los objetos. Así $abc$ y $bca$ son permutaciones diferentes.  Pero en algunos problemas, uno está interesado sólo en elegir objetos sin importar su orden.  Tales selecciones se llaman \emph{combinaciones}.  Por ejemplo, $abc$ y $bca$ representan la misma combinación.



El número de combinación $ C^{n}_{r} $ al elegir $r$ objetos de una colección de $n$ diferentes está dada por el \emph{número combinatorio}
\begin{align}
	\label{1.29}
	C^{n}_{r} = \comb{n}{r}=\dfrac{n!}{r!\left( n-r \right)!}
\end{align}



{Algunas fórmulas combinatorias}
\begin{align}
	\label{1.30}
	\comb{n}{r}&=\dfrac{P(n,r)}{r!} \\
	\label{1.31}
	\comb{n}{r}&=\comb{n}{n-r}\\
	\comb{n}{r}&=\comb{n-1}{r-1}+\comb{n-1}{r}
\end{align}


{}
\begin{ejemplo}
	\label{exmp:1.18}
	En una baraja inglesa, ?`cuantas formas hay de escoger dos cartas del mismo palo?
\end{ejemplo}


{}
\begin{thm}[Teorema del binomio]
	\begin{align}
		\label{1.32}
		\left( x+y \right)^{n} = \sum_{r=0}^{n}\comb{n}{r}x^{r}y^{n-r}
	\end{align}
	
\end{thm}


{Aproximación de Stirling}
\begin{align}
	\label{1.33}
	n! \approx \sqrt{2\pi n}\left( n^{n}e^{-n} \right)
\end{align}


