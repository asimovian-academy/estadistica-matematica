\section{Problemas}

\begin{problema}
	%schaum solved 1.15, 1.17
	\begin{enumerate}
		\item 
		La caja I contiene 3 canicas rojas y 2 azules, mientras que la caja II contiene 2 canicas rojas y 8 azules. Se lanza una moneda justa. Si la moneda sale cara, se elige una canica de la caja I; si sale cruz, se elige una canica
		de la caja II. Calcula la probabilidad de que se elija una canica roja.
		\item Supongamos que el que lanza la moneda no revela si ha vuelto cara o cruz (de modo que no se revela la caja de la que se eligió una canica), pero sí revela que se eligió una canica roja. ¿Cuál es la probabilidad de que se eligiera la caja I (es decir, que la moneda saliera cara)?
	\end{enumerate}
\end{problema}

\begin{problema}
	%wackerly ejemplo 2.23
	Un fusible electrónico es producido por cinco líneas de producción en una operación de manufactura. Los fusibles son costosos, sumamente confiables y se envían a proveedores en lotes de 100 unidades. Como la prueba es destructiva, la mayoría de los compradores de fusibles prueban solo un número pequeño de ellos antes de decidirse a aceptar o rechazar lotes de fusibles que lleguen.
	
	Las cinco líneas de producción producen fusibles al mismo ritmo y normalmente producen solo 2\% de fusibles defectuosos, que se dispersan al azar en la producción. 
	
	Desafortunadamente, la línea 1 de producción sufrió problemas mecánicos y produjo 5\% de piezas defectuosas durante el mes de marzo. 
	
	Esta situación llegó al conocimiento del fabricante después de que los fusibles ya habían sido enviados. Un cliente recibió un lote producido en marzo y probó tres fusibles. Uno falló. 
	
	¿Cuál es la probabilidad de que el lote se haya producido en la línea 1? ¿Cuál es la
	probabilidad de que el lote haya provenido de una de las otras cuatro líneas?
\end{problema}