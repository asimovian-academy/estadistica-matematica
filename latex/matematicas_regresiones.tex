\section{Entendiendo las matemáticas detrás de la regresión lineal}

Supongamos que tenemos una base de datos hipotética que contiene la información acerca del costo (en unidades de $\$10000$) de varias casas y sus respectivos tamaños (en pies cuadrados $ft^{2}$).


\begin{center}
	\begin{tabular}{|l|l|}\hline
		Tamaño & Costo\\\hline
		1500 & 45\\\hline
		1200 & 38\\\hline
		1700 & 48\\\hline
		800 & 27\\\hline
	\end{tabular}
\end{center}



En este caso, el costo es la variable de salida, mientras que el tamaño es la variable de entrada. 

La entrada y la salida generalmente se denotan por $X$ y $Y$, respectivamente.


En el caso de la regresión lineal, supondremos que el costo $Y$ es una función lineal de tamaño $X$ y para estimar $Y,$ proponemos el modelo \begin{align}
	Y_{e} = \a + \beta X ,
\end{align}
donde $Y_{e}$ es el \emph{valor estimado} de $Y$ con base en nuestra ecuación lineal.


\begin{observacion}
	El propósito de la regresión lineal es encontrar valores $\alpha, \beta$ estadísticamente significativos, que \emph{minimicen} la diferencia entre $Y$ y $Y_{e}.$
\end{observacion}



En el caso de nuestro ejemplo, si encontramos los valores de $\a=2$ y $\beta=0.03,$ entonces la ecuación será
\begin{align}
	Y_{e}= 2 + 0.03 X.
\end{align}


Usando esta ecuación, podemos estimar el costo una casa de cualquier tamaño.  Por ejemplo, para una casa de $900 ft^{2}$, el costo será
\begin{align}
	Y_{e}= 2 + 0.03(900)= 29 .
\end{align}


La siguiente pregunta que nos haremos es como estimar $\a$ y $\beta.$  Para esto usaremos un método llamado suma de  \emph{mínimos cuadrados} para la diferencia entre $Y$ y $Y_{e},$ que representaremos como
\begin{align}
	\ep = Y-Y_{e}.
\end{align}



Nuestro objetivo es minimizar
\begin{align}
	\sum \ep^{2} & = \sum \left( Y-Y_{e} \right)^{2}\\
	&= \sum \left( Y - \left( \a+\beta X \right) \right)^{2}
\end{align}
respecto de los parámetros $\a, \beta.$


Utilizando un poco de cálculo, se puede demostrar que los valores de los parámetros que minimizan la suma anterior son
\begin{align}
	\label{beta}
	\beta &= \dfrac{\sum \left( x-\bar{x} \right)\left( y-\bar{y} \right)}{\sum\left( x-\bar{x} \right)^{2}}\\
	\label{alfa}
	\a &= \bar{y}-\beta\times \bar{x}
\end{align}

% \part{NO BORRAR}
% \section{Recursos}
% \subsection{analisisPredictivo}
% El repositorio con los scripts de \texttt{Python 3} de esta presentación los puede encontrar en \href{https://github.com/julihocc/ulsaPye/tree/master/analisisPredictivo}{https://github.com/julihocc/ulsaPye/tree/master/analisisPredictivo}
% 
% [t, ]
% \frametitle{Referencias}
% \nocite{*}
% \betaibliographystyle{amsalpha}
% \betaibliography{ulsaPye}
% 